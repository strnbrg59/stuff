\twelvepoint
\def\interline{12pt}
\font\bfsl=cmbxsl10 at 12pt

%**** this is for consecutively-numbered footnotes...
\newcount\notenumber
\def\clearnotenumber{\notenumber=0\relax}
\def\note#1{%
  \advance\notenumber by 1\baselineskip=12pt
  \footnote{$^{\the\notenumber}$}{{\tenpoint #1\vskip -12pt}}
  \baselineskip=\interline}
\clearnotenumber
%*********************

\def\section#1{\vskip 4pt\noindent
  {\baselineskip 9pt\relax\bf#1}\vskip 0pt\noindent}
\def\subsection#1{\vskip 4pt\noindent
  {\baselineskip 9pt\relax\bfsl#1}\vskip 0pt\noindent}

\parskip=4pt
\baselineskip=\interline
\hsize=6.5 true in
\vsize=8 true in\relax
\hoffset= -.05 true in
\voffset=.5 true in

\headline{\tenpoint\sl BA130---Financial Management\hfil T.D.\thinspace Sternberg}

\centerline{{\bf The Market Game, round 3}}

\section{I. What's new}%
There is no firm and no stock this time.  Instead, we will trade
bonds, in the process endogenously determining the term structure of
interest rates.  The purpose of this round is for you to learn how
interest rates are determined in a free market, and how you can
enrich yourself trading in that market.

\section{II. Bonds}%
We will trade three bonds.  One is a zero-coupon bond that matures in
one week and pays \$10; we will call this the {\it bill}.  The other
bonds are coupon bonds.  One of them pays \$2 at the end of the first
week and \$10 at the end of the second: we'll call this {\it bond2}.
The other coupon bond pays \$8 and \$10: we'll call this one {\it bond8}.

In previous rounds of the game, Fed fixed the price of the one bond to
reflect a constant per-week interest rate.  This time, Fed steps aside
and lets you determine the whole term structure of interest rates 
through your trading.
				   
\section{III. Money and forced sales}%
Money plays the same role as in the past.  You use money to pay for
securities you buy, while selling securities adds to your money
holdings.  The objective is, as always, to have as much money as
possible at the end of the game.  As before, the program will not let
your money holdings go negative.  To prevent that, forced sales will
take place in the bill market.

\section{IV. Initial endowments}%
You start with \$50 cash and no securities.

\section{V. Strategy}%
If you plan to play only a little, at least don't let your cash sit
idle; buy some bonds and earn interest.  Beyond this, there are three
more ways to make money---arbitrage, the bid-ask bounce, and
speculation.

\subsection{A. Arbitrage}%
An arbitrage opportunty awaits you any time you can buy one of the
three securities for less than you can sell its equivalent in terms
of a combination of the other two.  You got some practice on the
problem set.

Just remember that because of the bid-ask spread, you cannot buy and
sell securities at the same price.  Therefore in calculating
arbitrage trades, be sure you take account of which prices are asks
and which are bids.  For a really fast buck, execute arbitrage trades
with market orders.  On the other hand, you'll get better prices with
limit orders, but your orders may not all execute.

\subsection{B. Bid-ask bounce}%
The idea is to place limit orders so you repeatedly sell at the ask
and buy at the bid, thereby earning the bid-ask spread every ``round
trip''.  In past rounds of the game, this was not a significant source
of income because you were limited to a very small number of orders.
This time, however, you may place up to 100 orders each week in each
security, a limit you will find hard to run up against since you still
have only three hours of connect time per week.  Thus, you now have 
an opportunity to be a real market maker and make real money at it.

The market orders that will generate your bid-ask bounce income will
come only in part from your classmates.  We have a new player this
round, Poisson.  Poisson is a computer program that places random
market orders all day long.  See section VI for details on Poisson's
behavior.

\subsection{C. Speculation}%

\subsection{\quad 1. What determines the interest rate!?}%
This round you're not speculating about Dow Jones index changes, but
about the interest rates in our game.  While a natural first reaction
is consternation and confusion over the apparent indeterminacy of our
interest rates, on second thought there really is a great deal to go
on.

Start by ignoring the bonds; imagine our game ran for just one week,
and that the bill were the only security.  This way we can talk about
just one interest rate, and focus on the essentials.

Our first insight is that the interest rate will not be negative; no
one will lend at a negative interest rate because money offers an
interest rate of zero.

Having established that the interest rate will be at least zero
percent, let's consider why it might actually be positive.  The
interest rate is the rental price of money.  Now, who would want to
rent money?  The answer is, anyone who hopes to be a market maker
i.e. earn the bid-ask bounce.  Your friends the market makers will
need a stockpile of money in order to avoid forced sales when their
limit buy orders cross with market sells.  For example, suppose a
market maker has only \$4 money, and her limit buy at \$9 crosses with
a market sell.  The trade drops money to --\$5, which instantly
generates a forced sale on behalf of the market maker.  Result: the
market maker earned the bid-ask spread on the first trade, then lost
it back on the forced sale.%
\note{In fact, it's even worse for the market maker, because the forced
sale will cross with the best limit buy order {\it below} hers on the
order book.}

For a market maker, then, a money stockpile is working capital.  The next
question to ask is, how high an interest rate would a market maker pay
for this working capital?  The market maker will look at how productive
the working capital will be, which means how much income off Poisson
it will support.  A small amount of working capital---say \$20---could
pretty easily be parlayed into profits approaching 100\% per day.
A larger amount of working capital will support profits larger in dollars
but smaller as a percent of working capital.  Market makers will look
at the interest rate and borrow money up to the point where the marginal
productivity of the money equals the interest rate.  So, if the interest
rate is low, market makers will borrow a great deal of money; if the
interest rate is high they will borrow less.

That, in other words, sums up the demand for money!  As for the supply
of money, it's just a vertical line (i.e. infinitely inelastic) at the
sum of everyone's initial endowments, plus whatever dribbles in at
Poisson's expense.  Ultimately, the interest rate(s) will be whatever
clears the ``rental'' market for money and part of your challenge to
develop a feel for that.


\subsection{\quad 2. How to speculate}%
Suppose you think the short interest rate will be lower tomorrow.
Then you should buy bills.  Or you can place a bet on the long
interest rate by trading combinations of securities equivalent to a
two-week zero-coupon bond.  In fact, bond2 has such a low first-week
coupon that it approximates a two-week zero well enough if your
purpose is just to bet on the long interest rate.  To bet on the
future spot rate, put together a forward contract.

You may find yourself speculating against your will if your attempt
to be a market maker results in a large long or short position: be
careful.

\section{VI. Poisson, the noise trader}%
A fictional character named Poisson (actually a computer program
running out of my account) will place random market orders daily from
8AM to 6PM, while the game is up and running.  Poisson provides an
important element of real markets that would otherwise be missing from
our game---noise trading.  A noise trader (in the real world) is a
person who trades for other than speculative motives, or speculates
but does so in an unintelligent fashion so as to appear random.  The
principal nonspeculative motive for trading is consumption and saving;
people sell securities when they want to finance a major purchase, and buy
securities to save.  Seen in aggregate, the
behavior of these traders appears random, hence the term {\it noise
traders.}  Noise traders are an important feature in the real
world, so it's time we simulated them in the game.

Poisson's trading will benefit you.  The benefit is that if you have
a good bid or offer on the limit order book, you can expect it to
cross even if none of your classmates think you're being particularly
generous.  Poisson's average order size is 10 units,%
\note{Poisson will be active in all three markets.}
alternating randomly between buys and sells, and he trades about
every five minutes.  The process of buying high (at the ask) and
selling low (at the bid) ensures Poisson will lose money to those of
you with the most competitive bids and asks.

Poisson will not trade if the bid-ask spread is wider than 10\% of the
lowest ask; he's random but not stupid.

\section{VII. Fed}%
Fed will try hard to stay out of the action.  Nonetheless, someone has
to make sure the bid side of the bill order book never empties out
entirely (for if it did, forced sales would fail to clear).  Fed will
therefore place large ``backstop'' orders at prices so ridiculously low
they should always be below bids from real players.  Hopefully, the
backstop orders will never get hit.

\section{VIII. Grading}%
One-half of your grade rides on your performance in the game.  This
is based, as before, on your final wealth in standard deviations from
the mean.  There will, however, be a slight change to the way that
standard deviation and mean are calculated: they will be calculated
after I discard the top and bottom 5\% of the distribution. This
should result in a more equitable and sensitive grading scale which
will not be unduly influenced by a few people who realize
extraordinarily large gains or losses.

A write-up will determine the other half of your grade.  I will
announce details of the write-up later.  For now, to ensure you will
have something to write about, just make sure you clear at least one
trade in each market, in each week.

\section{IX. What happens at the end of the first week}%
At the end of the first week, the bill and bonds pay their first (and
in the bill's case last) coupons.  At the same time, the order books
are reduced by the amounts of the coupons (just as in the stock
markets).  The bill market shuts down, with subsequent forced sales
routed to the bond8 market.  

\section{X. Market manipulation and irresponsible play}%
With as many as 300 orders at your disposal, you can do yourself alot
of damage.  Wild, irresponsible play, or bald attempts to manipulate
prices, will also detract from your classmates' learning experience.
For these reasons, I reserve the right to shut down the account of
any person who I deem to be out of control, or so deep in the hole as
to have lost any incentive to play responsibly.

\section{XI. Timetable}%
\medskip
\settabs 2 \columns
\+7AM, Tuesday April 26		&game begins\cr
\smallskip
\+8AM -- 6PM daily		&Poisson trades in all three markets\cr
\smallskip
\+7AM, Tuesday May 3            &bill pays \$10\cr
\+		                &bond2 pays \$2\cr
\+				&bond8 pays \$8\cr
\smallskip
\+7AM, Tuesday May 10		&trading ends\cr
\+				&bond2 pays \$10\cr
\+                              &bond8 pays \$10\cr

\bye
