\input 12point
\twelvepoint
\def\interline{12pt}
\font\bfsl=cmbxsl10 at 12pt

%**** this is for consecutively-numbered footnotes...
\newcount\notenumber
\def\clearnotenumber{\notenumber=0\relax}
\def\note#1{%
  \advance\notenumber by 1\baselineskip=12pt
  \footnote{$^{\the\notenumber}$}{{\tenpoint #1\vskip -12pt}}
  \baselineskip=\interline}
\clearnotenumber
%*********************

\def\section#1{\vskip 4pt\noindent
  {\baselineskip 9pt\relax\bf#1}\vskip 0pt\noindent}
\def\subsection#1{\vskip 4pt\noindent
  {\baselineskip 9pt\relax\bfsl#1}\vskip 0pt\noindent}

\parskip=8pt
\parindent=0pt
\baselineskip=\interline
\hsize=6 true in
\vsize=8 true in\relax
\hoffset=.3 true in
\voffset=.5 true in

\headline{\tenpoint\sl Business S232---Financial Markets and Institutions\hfil T.D.\thinspace Sternberg}

\centerline{{\bf The Market Game: IPO Session}}

\section{I. Introduction}%
You are assigned to the board of one of two firms.  You and the other
members of your board are in charge of choosing an investment policy,
and arranging an IPO to raise the required cash.  

\section{II. The firm's investment opportunities}%
For your firm's investment, it may buy ``Dow Jones days''.  If it
buys just one day, earnings will be the square root of the largest
one-day percentage change in the DJIA during the week,%
\note{For a precise definition of the ``week'', and details on all
other timing issues, see the last section of this handout.  We're
taking square roots in order to reduce risk; thus 0.3\% becomes 0.55\%,
2\% becomes 1.4\% and so on.}
times a constant $k$.  Two days buys the sum of the square roots of
the two largest daily percentage changes times $k$, and so on up to
five ``Dow Days''. $k$ is \$1000 for firm 1, \$2000 for firm 2.  If
your last name begins with the letters A--L, your firm is firm 1.
Otherwise your firm is firm 2.

You may think of these firms as operating in the same business, but
enjoying different levels of productive efficiency.  Notice that
every additional day adds a smaller amount to revenues.  That's
diminishing returns.

Investing isn't free, however.  Each Dow Day costs your firm \$400,
payable up front.

\section{III. The IPO, or how to finance your firm's investment}%
Meeting with your fellow board-members, you will decide how many Dow
Days to invest in.  Having decided on say, $n$ days, you will need to
raise \$400$n$.  In the past, your firm has relied on the credit
markets.  Indeed, it is already saddled with \$300 in debt; your firm
is short $30$ bonds.  Further borrowing seems unwise, and so you are
going to raise the \$400$n$ you now need with a stock issue---an IPO.

The issue price of this IPO is the other decision you will make.
Pre-IPO, your firm has 160 shares outstanding (16 people $\times$
10 shares per capita).  Once you decide how many Dow Days to buy,
the IPO problem becomes one of choosing an issue price $p$ and the
number of shares to issue $N$.  You want $p$ and $N$ to satisfy two
conditions: $pN$ must equal the cost of your desired Dow Days, and
$p$ must be low enough to induce people to pay it.  Of course, you
don't want to set $p$ too low either; your endowment of 10 shares
is part of your personal wealth.

Your investment decision and issue price will surely be affected by
the interest rate, i.e. the yield on our bonds.  To relieve you of
the burden of predicting future interest rates, the bond market will
open two days before you have to decide on the IPO.  Two days should
be plenty for the bond market to settle down and thus provide you
with a good reading on the interest rate.%
\note{Of course, if you choose to participate in the ``pre-issue''
bond market, you will have to think about credit conditions in the
future.  To this end, please bear in mind that much, perhaps most,
of the money supply will disappear at the moment the Dow Days are
paid for.}

In accordance with your board's decision on $p$ and $N$, I will place
a limit sell order in your firm's name.  For example, if you wish to
raise \$1600 with an issue of $160$ shares priced at \$10 apiece, I
will place a limit order that will look like this:

\medskip
\settabs 6 \columns
\+&&&offer\cr
\+&&{\it firmname}&160&10.0\cr
\medskip

With this offer on the limit order book for your firm's stock,
everyone is free to snap up the IPO via market buy orders.%
\note{The computer will reject limit orders to sell at prices below
your IPO's, so you need not worry about bad people interfering with
your firm's fundraising effort.}
Payment for your Dow Days is due 36 hours after the IPO begins.
Therefore should your IPO fail to sell out, your firm will be in
serious trouble, as it will be unable to invest in all the Dow Days
planned, and the \$300 debt will still be there.

You'll have to think hard about the offering price, and of course
you're free to encourage people from the other firm to buy
your stock.  Effective selling is an important part of investment banking.%
\note{Part of your marketing might include a money-back guarantee, to
the effect that if your firm fails to raise some minimum amount of
money in its IPO, it will refund all the money it did
raise.  (The computer can carry out such a refund in a way that
causes the refundees no inconvenience.)  Without such a guarantee,
you might find everyone waiting for ``everyone else'' to take the
firm's offer first, knowing that if the offer fails the firm will
suddenly be worth much less.  Another tactic you could try is
a two-tiered offer; offer some stock at a low price, and the rest at
a higher price, hoping thereby to create a stampede to pick up the
lower-priced part of the offering.  The prospectuses (option `p') will
contain information about these details of the offering.}

\section{IV. Dividends and limited liability}%
As a corporation, your firm enjoys limited liability.  Therefore, in
the event of an unsuccessful IPO or a disastrously quiet week at the
NYSE, your firm may default on its \$300 debt.  In that case, the
dividend will be zero.  Negative dividends are impossible.

Otherwise, here is an example of how the dividend would be calculated
for firm 1.  Suppose this firm invests in three Dow Days, and the
changes in the Dow turn out to be 0.9\% on Friday, 0.1\% on Monday,
0.7\% on Tuesday, 0.2\% on Wednesday, and 0.14\% on Thursday.  Then
this firm pays a dividend of
$$
\$1000(\sqrt{0.9}+\sqrt{0.7}+\sqrt{0.2}) - \$300 = \$1933,
$$
which is then divided by the number of outstanding shares. This firm
would default only if the three biggest Dow Days added up to less
than $0.3^2$\%.

\section{V. The securities}%
You start out with \$150, ten shares {\it in your firm only},
and no bonds.%
\note{Those \$300 debts the firms all carry are not owed to anyone in
the class; from your point of view, the \$300 will just disappear at the
end of the game.}

Bonds pay \$10 at the end of the game, and the interest rate is
endogenous, as before.

There will be two stock markets, one for each firm's stock.  As an
investor, you may invest in either stock.

Your firm comes into the world with no cash, negative 30 bonds, and
as many negative shares as the number of people on your board (16),
times ten.  After the IPO, your firm will have more shares
outstanding. If you are ever in doubt as to the balance sheet of
either firm, just ``finger'' it (option `f' in the main menu); the
firms are named firm1 and firm2.

\section{VI. Limits on how much you may play}%
You are limited to three hours of connect time.  Connect time is how
long you are inside the aqc program. ``Finger'' your knickname to
find how much time you have left.

\section{VII. Grading}%
You start at a B+ and gain (lose) one-third of a grade for every
standard deviation your final wealth is above (below) the class mean.
To ensure fairness across firms, your final wealth will be adjusted
for the ex-post value of your initial endowment of stock.%
\note{Only after this adjustment will it become meaningful to compare
the wealth of people who started out with ten shares of different
firms.}

This ex-post value needs defining.  At the end of the game I will
calculate how many Dow Days would have maximized the value of your
firm.  The ex-post value of your endowment will then be defined as
what your ten initial shares {\it would} have been worth at the end
of the game had your firm invested in the ex-post optimal number of
Dow Days and financed that investment with an issue of stock priced
at the ex-post value per share.  Thus you will be competing against
members of your firm's board as well as against the other firm's.  If
the Dow does something surprising, it will probably surprise both
boards in the same direction and hence not hurt your standing {\it
vis-a-vis} the other board.  What will hurt you, however, is
mispricing or mismanaging your IPO.

\section{VIII. Poisson}%
Poisson will be active in all three markets.  His hours are 6AM to
10PM seven days a week.  In each market, Poisson trades about every
five minutes, placing a market order that averages about 10 stocks
or bonds.  

\section{IX. Fed}%
Although the interest rate is endogenous, we need a lender of last
resort for emergencies.  Accordingly, Fed will place large bids (but
no offers) reflecting a 200\% per week interest rate.

\section{X. Market manipulations}%
I discourage attempts to manipulate prices.  In particular, you may
not hold more than \$1000 in cash at any time.

\section{XI. Timetable}%
\settabs 2 \columns
\+Monday 7/18, 9PM	&Trading opens in the bond market.\cr
\+Wednesday 7/20, in class &Decision on investment and IPO pricing.\cr
\smallskip
\+Wednesday 7/20, 9PM	&IPOs appear on the limit order books.\cr
\+			&Trading opens in the stock markets.\cr
\smallskip
\+Friday 7/22, 9:00AM  	&Unsold part of IPO offers are cancelled.\cr
\+			&Cost of Dow Days deducted from firms' accounts.\cr
\smallskip
\+Fri 7/22 -- Thurs 7/28  &Dow Days; your firm is credited with the \cr
\+			&largest $n$ of these ($n\leq 5$).\cr
\smallskip
\+Thurs 7/28, 1PM	&Game ends.\cr
\+			&Stocks pay dividends.\cr
\+			&Bonds pay \$10 apiece.\cr

\bye