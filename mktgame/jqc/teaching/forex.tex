\twelvepoint
\def\interline{12pt}
\font\bfsl=cmbxsl10 at 12pt

%**** this is for consecutively-numbered footnotes...
\newcount\notenumber
\def\clearnotenumber{\notenumber=0\relax}
\def\note#1{%
  \advance\notenumber by 1\baselineskip=12pt
  \footnote{$^{\the\notenumber}$}{{\tenpoint #1\vskip -12pt}}
  \baselineskip=\interline}
\clearnotenumber
%*********************

\def\section#1{\vskip 4pt\noindent
  {\baselineskip 9pt\relax\bf#1}\vskip 0pt\noindent}
\def\subsection#1{\vskip 4pt\noindent
  {\baselineskip 9pt\relax\bfsl#1}\vskip 0pt\noindent}

\parskip=4pt
\baselineskip=\interline
%\hsize=6.5 true in
\vsize=8 true in\relax
\hoffset= -0.000125 true in
\voffset=.5 true in

\headline{\tenpoint\sl BA132---Money and Capital Markets\hfil T.D.\thinspace Sternberg}

\centerline{{\bf The Market Game, round 5}}

\section{I. What's new}%
Two currencies, a foreign exchange market, and endogenous interest rates.
This round lasts just one week.  Also, there have been changes to Poisson's
hours and some other key times---see the timetable below.

\section{II. Currencies}%
Until now, money meant Dollars.  This time, we have Pounds as well as
Dollars.  You may trade Pounds for Dollars and Dollars for Pounds on a
market that works just like the markets you're accustomed to---there
will be market and limit orders to buy and sell Dollars, where the
prices are denominated in Pounds.

\section{III. Bonds and interest rates}%
We will trade two bonds.  The {\it dbond} pays \$10 at the end of the
week.  The {\it pbond} pays {\it\$}10 at the end of the week.  The
market prices of these bonds will determine the interest rate for
Dollars and Pounds respectively.

\section{IV. Initial endowments}%
You start with \$400 and {\it\$}100, and zero units of each bond.

\section{V. Objective}%
As always, you wish to become richer than your classmates.  I'll
define your final wealth as the sum of your Dollars and your Pounds,
that is the final exchange rate will be 1.00.  (However, until then
the exchange rate will be determined in the market.)

\section{VI. Strategy hints}%
Have you ever wondered how exchange rates and interest rates
influence one another?  It's one of the most important problems in
international finance.  And although no one has figured it all out
yet, when the experts sit down to think about this problem, they
usually think in terms of a model that resembles the set-up in this
round of the market game.

First, forget about the exchange rate.  That leave two markets---the
Pound bond and the Dollar bond---you'll have no trouble with by now.
You know interest rates depend on money supplies, and consequently
the Pound interest rate will be greater than the Dollar interest rate
(that is, the pbond will be worth less than the dbond).

Now let's bring in the exchange rate.  Say the game is on and you see
the Pound interest rate is higher than the Dollar interest rate.  And
say the exchange rate is 1.00.  Then you should hurry up and sell
Dollars; every Dollar you sell yields you one Pound, which you invest
at the higher interest rate.  In fact, you shouldn't settle for
converting only the Dollars you have, but rather you should actually
borrow Dollars (by selling dbonds) to finance those Pound purchases.

This is better than speculation; it's an honest-to-goodness
arbitrage, because you {\it know} that the end-of-game exchange rate
will be 1.00.  But like other arbitrage opportunities, it will
disappear as investors exploit it.  As people sell Dollars and buy
Pounds, they cause the Dollar to devalue, say to where a Dollar will buy
only {\it\$}0.75 or even less.  This arbitraging also influences the
interest rates; demand for Pound bonds reduces the Pound interest rate,
while supply of Dollar bonds increases the Dollar interest rate.

\section{VII. New features for the aqc program}%
From the main menu, you choose dbonds, pbonds, Dollars or Pounds.
Choosing dbonds or pbonds drops you into the same security-level menu
you're familiar with: just remember that dbond transactions are done
in Dollars, while pbond transactions are done in Pounds.

Choose Pounds from the main menu and you will see the display that until
now you've obtained from choosing the `\$' option, i.e. a list of your
cash holdings, showing how they've been affected by your trades.

The only really new feature is the Dollar menu.  Where in the past
Dollars were just cash, now they are both cash and a security.  When
you choose Dollars from the main menu, you'll see a security-level
menu just like the one for dbonds or pbonds.  So you'll be able to
place orders in what appears to be a ``Dollar'' market.  This is indeed
a market for Dollars---Dollars with prices in terms of Pounds!  When you
buy Dollars in this market, you're simultaneously selling Pounds (and
you can see the effect by returning to the main menu and choosing the
Pounds option).  The Dollar menu offers you all the familiar
options---place or cancel an order, look at the ticker, look at your
account history, etc.---plus one more, option `\$', which works like
`\$' has worked in the past, i.e. it shows how your Dollar holdings
have developed over time.

I hope this hasn't scared you away; log in and you'll catch on
quickly.

\section{VIII. Poisson}%
Poisson will revert to his old 8AM--5PM schedule.  In the dbond
and pbond markets, Poisson's average trade size will be 3 bonds.  In
the Dollar market, Poisson will average \$30 per trade.  Every couple
days, I'll let you know how much money---Dollars and Pounds---Poisson
has injected into the game.

\section{IX. Maximum/minimum actions you may/must take}%
You may take no more than 28 actions in each market.  The minimum, to
be considered for a passing grade, is just to do something with your
cash holdings within 48 hours (preferably, to spend them all on bonds,
unless you have something more profitable in mind).

\section{X. Limits on trade sizes}%
The largest market order you may place is for 50 units of any
security.  However, in the Dollar market you may order up to \$500 at
one time.

\section{XI. Position size limits}%
In order to prevent manipulations of the interest rates, I'm going to
enforce a limit of \$4000 or {\it\$}1000 on the amount of cash you may hold.
In the bond markets, there's no limit whatsoever to what you may own,
and you need not cover short positions at week's end.

\section{XII. Forced sales}%
When you're short Dollars, the program places a forced sale for you in
the dbond market.  When you're short Pounds, the forced-sale market is
the pbond market.  Trades in the Dollar market are clearly capable of
triggering either kind of forced sale.

\section{XIII. Price floors and ceilings}%
Prof will cap the bond markets at 10.1 on the upside, and on the downside
at a price consistent with a 200\% per-week interest rate.

\section{XIV. Timetable}%
\medskip
\settabs 2 \columns
\+5PM, Tuesday November 2	&game begins\cr
\smallskip
\+8AM -- 5PM daily			&Poisson trades\cr
\smallskip
\+10AM, Tuesday November 9  &trading ends\cr
\smallskip
\+5PM, Tuesday November 9	&dbond pays \$10\cr
\+		                	&pbond pays {\it\$}10\cr
\+							&final wealth = \$ + {\it\$}\cr

\bye