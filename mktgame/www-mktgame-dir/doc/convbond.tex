\input workpap
\parskip=4pt
\parindent=0pt
\hsize=6.25 true in
\vsize=8 true in\relax
\hoffset=.1 true in
\voffset=.5 true in

\headline{{\tenpoint\sl BA133---Investments\hfil T.D.\thinspace Sternberg}}
\footline{\hfil\folio\hfil}

\centerline{{\bf The Market Game, Round 2}}

\section{I. What's new}%
Convertible corporate bonds: you have your choice between a \$10
end-of-game payoff, on the one hand, and a share of stock, on the
other.  Our other securities are stock, a call option on the stock,
and the kind of risk-free bonds we had last time.

\section{II. Stock}%
As in round 1, everyone starts with ten shares.  The firm's earnings
are also the same as in round 1---\$10,000 times the sum of the
absolute values of the percent changes in the Dow Jones Industrial
Average over five days.

Now here are the differences.

\subsection{a. Fewer shares outstanding} 
We are down to seventy-six
students, so there are initially 760 shares outstanding.

\subsection{b. Bond claims are senior}
There are also 760 convertible bonds outstanding, and they have to be
paid off before there can be any dividend.  As in the real world,
shareholders are the {\it residual claimants}; they get everything
left after the more senior claims are paid.

\subsection{c. Bond conversion dilutes stock}
Holders of convertible bonds may trade in (``convert'') their bonds
one-for-one for shares.  Therefore, if all the bonds get converted,
there will be 760 more shares to spread the dividend over.  (Of
course, there would be more dividend to go around then, since
converted bonds have no more claims as bonds; conversion extinguishes
them.)

Here is an example.  Suppose the sum of the absolute percent changes
in the DJIA is 7.00.  Earnings then are \$70,000.  If none of the
bonds are converted, \$7600 goes to bondholders and the remaining
\$62,400 gets divided up among the 760 shares, for an \$82.11
dividend per share.

In such a situation, however, the bonds will be converted because the
dividend is greater than the \$10 payoff to unconverted bonds.  With
all the bonds converted, the entire \$70,000 is available as a
dividend---unfortunately spread over 1520 shares.  That makes the
per-share dividend \$46.05.

This example has assumed a total DJIA change of 7\%, which is rather
high.  At more moderate levels of DJIA volatility, the convertible
bonds will remain unconverted.  For example, if the total DJIA change
is 1.00\%, the dividend per share will be
$$
{10,000 - 760\times10\over 760} = \$3.16.
$$

A really tranquil week will leave no money at all for dividends, and
result in a partial bond default.  Suppose the DJIA moves by just 0.5\%.
That makes earnings a mere \$5000, which is short of the bonds' \$7600
face value.  Each bond then receives \$6.58, and the shares get nothing.

Do you remember when I tried to sound intelligent by saying stock is
a call option on the firm's assets?  Now you know what I meant.%
\note{Almost.  Bond conversion dilutes the shares.  So to be precise,
our stock amounts to a {\it combination} of call options: carefully
draw a payoff diagram and you'll see.}

\section{III. Convertible Bonds}%
You start with ten convertible bonds.  Each one entitles you to \$10
or the firm's earnings divided by the number of outstanding
convertible bonds, whichever is less.  Optionally, you may convert
each bond you own into a share of stock.  In that case, your
converted bond disappears and in its place you own one share of
stock.  The conversion option is a European call on the stock, with
``exercise price'' equal to the bond itself.%
\note{In the real world, it's common for the conversion option to
yield more or less than exactly one share.  Our game avoids this
superfluous complexity.}

Since the convertibility option is European, you won't see the `e'
option.  I will see to converting all the bonds, at the end of the
game, if and only if the dividend exceeds \$10.  Thus at the end of
the game, there will be either 760 convertible bonds, or no
convertible bonds.

For some examples, reread section II above.

\section{IV. Call options}%
These are American call options on the stock, with exercise price
\$10.  I will exercise your unexercised in-the-money options at the
end of the game.  In that case, you will pay the exercise price to
receive whatever dividend each share of stock has coming to it.

The firm has no involvement with these call options.  Therefore,
exercise does not affect the number of shares.

\section{V. Treasury Bills}%
These are what we called ``bonds'' last time.  (I've changed the name
to avoid confusion with the convertible bonds.)  As before, Fed will
rig this market so the prices reflect a 100\% per week interest rate.


\section{VI. Initial endowment} %
You start with \$100 of money, ten shares, ten convertible bonds,
no call options, and no Treasury Bills.

\section{VII. Limits on size and number of orders}%
As before, you may place no more than ten orders (market plus limit)
in the ``free'' markets---stocks, convertible bonds, and call
options.  You may place up to 100 orders in the Treasury Bill market.
See the handout for the first market game (under section VII), if you
aren't sure about how I count orders placed.

The largest market order you may place is one for fifty units of any
security.  The largest limit order you may place is one for 150 units
of any security.  

\section{VIII. Grading}%
Grading works the same as last time.%
\note{You start at a B+.  For every 1/2 standard deviation above
(below) the mean you gain (lose) 1/3 of a grade.}

As an added incentive for rational play, I hereby pledge to include,
in the final exam, the following questions that relate to this round
of the game.  ``Draw the payoff diagram for the stock.''  ``What
combination of call options would be equivalent to a share of
stock?''  ``Compare the convertibility option to the simple call
option; should they have been worth the same?''  ``Draw a binomial
tree to value the stock.  Explain how you would proceed.''  ``Draw a
binomial tree to value the call option.''  ``Draw a payoff diagram for
the convertible bond.''

\section{IX. Strategy hints}%
First, review the strategy hints from last time; they all apply.
Beyond that, think about the exam questions above.  James and I will
be glad to help you grapple with them.

\section{X. Poisson}%
Our noise-trader friend will continue as before, trading in each market.

\section{XI. Timetable}%
All times are Pacific coast times.
\medskip
\settabs 2 \columns
\+11AM, Friday October 28     &trading begins.\cr
\smallskip
\+1PM Friday--Thursday   &NYSE trading ends, closing DJIA is determined.\cr
\smallskip
\+1PM, Thursday November 3    &Treasury Bills pay \$10.\cr
\+                            &All in-the-money options are exercised.\cr
\+                            &Unconverted convertible bonds pay \$10.\cr
\+                            &Shares pay dividends.\cr
\+                            &Your final wealth is your final money.\cr
\+                            &Game ends.\cr

\bye

