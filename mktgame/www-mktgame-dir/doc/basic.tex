\twelvepoint
\def\interline{12pt}
\font\bfsl=cmbxsl10 at 12pt

%**** this is for consecutively-numbered footnotes...
\newcount\notenumber
\def\clearnotenumber{\notenumber=0\relax}
\def\note#1{%
  \advance\notenumber by 1\baselineskip=12pt
  \footnote{$^{\the\notenumber}$}{{\tenpoint #1\vskip -12pt}}
  \baselineskip=\interline}
\clearnotenumber
%*********************

\def\section#1{\vskip 4pt\noindent
  {\baselineskip 9pt\relax\bf#1}\vskip 0pt\noindent}
\def\subsection#1{\vskip 4pt\noindent
  {\baselineskip 9pt\relax\bfsl#1}\vskip 0pt\noindent}

\parskip=4pt
\baselineskip=\interline
\hsize=6.25 true in
\vsize=8 true in\relax
\hoffset=.1 true in
\voffset=.5 true in

\headline{\tenpoint\sl BA133---Investments\hfil T.D.\thinspace Sternberg}

\centerline{{\bf The Market Game, round 1}}

\section{I. Introduction}%
You have been handed 10 shares of stock, \$100 cash and no bonds.  By means of
a computer program, you may trade stock and bonds with your
classmates.  Your goal is to end the game as wealthy as possible.

\section{II. The stock}%
Your stock, together with that of all your classmates, constitutes
the equity of a certain firm.  That firm has the following
characteristics.  It exists for just under three days---6PM Monday July
11, through 1PM Thursday July 14.  It earns income once---at 1PM
Thursday---and pays it all out (immediately) as a dividend.  The
dividend equals \$1000 times the sum of the absolute values of the
percent changes in the Dow Jones Industrial Average over the three
NYSE trading days Tuesday, Wednesday and Thursday.

To make that definition clearer, here is an example.  Suppose the
DJIA closing prices turn out to be as follows.

\bigskip
\settabs 5 \columns
\+& Day         &   DJIA      & \% chg (absolute value) &\cr
\+& 7/11        &  3642.11    &  --- &\cr
\+& 7/12        &  3624.60    & 0.48 &\cr
\+& 7/13        &  3676.71    & 1.44 &\cr
\+& 7/14        &  3665.25    & 0.31 &\cr
\bigskip

If these turn out to be the closing Dow prices, our firm would pay
out a dividend of
$$
\$480+\$1440+\$310 = \$2230.
$$
To find the dividend
per share divide these numbers by 440 (44 people enrolled in the
course times 10 shares per person).%
\note{I won't let add/drops during the time the game is on affect the
number of shareholders.}

That determines how much each of your shares is worth.  But you
didn't know what that was going to be, when the game began.  And
that's how it is with real stocks in the real world; they're valuable
because they pay dividends, but the dividends aren't known in
advance.  They can be guessed at and forecasted with varying degrees
of accuracy and sophistication, but there is always some uncertainty.
This uncertainty gets slowly resolved over time, and that is the main
reason stock prices fluctuate.  Every day (indeed every hour) some
new information arrives and you are able to improve your prior
estimate of the firm's earnings.  Likewise in the real world: every
day brings news that helps stock analysts improve their assessments
of how profitable a firm will be this quarter, this year, and beyond.
In both markets---ours and the real world's---news will cause stock
prices to change.%
\note{The most obvious dissimilarity with the real world is our
firm's unusual ``production function''; in the real world, firms
produce hamburgers or haircuts, not ``dow days''.  Alas, it's
impractical to produce real things in class!  So instead we turn to
the DJIA to essentially generate random news and cash flows for us.
The advantage of using the DJIA rather than a computer-generated
stream of random numbers is that the DJIA is a much-watched financial
index and you will benefit from the experience of being a Dow-watcher
for a while.}

In the real world, when two people differ in the value they place on
a share of stock, they will want to trade.  Stock exchanges such as
the NYSE, the AMEX, and the NASDAQ facilitate such trades.  In our
class, you may trade your stock by means of a computer program.

\section{III. Bonds}%
You don't have to invest all your money in stocks; we have two
securities in this game, and the other one is a bond.  Unlike the
stock, the bond's payoff is not risky.  Every bond you own will pay
you \$10 at the end of the game---guaranteed.

Unlike in the introductory round, the bond market will not be rigged!
You are free to buy and sell bonds; presumably you will buy bonds when
you believe them to offer a greater risk-adjusted return than stocks.

Equilibrium in the bond market will determine the interest rate in
our economy.  Thus, suppose you observe at 1PM on Tuesday July 12
that the price of bonds is \$5.79.  There being two more days to go
until the \$10 payout on the bonds, the interest rate is 
$$
\big((10/5.79)^{1/2} - 1\big)\times 100\% = 31.42\%
$$
per day.

\section{IV. Money (Dollars)}%
Money here is like cash in your pocket; it earns no interest.  Its
role is that of a means of exchange.  Anytime you buy stocks or
bonds, their price is deducted from your money holdings.  Sell stocks
or bonds and your money increases.%
\note{Of course, you can't get rich just selling stocks and bonds,
because while the sales will increase your money holdings, you will
owe money on the stocks and bonds you sold.  Just as you receive \$10
for every bond you own at the end of the game, you will pay \$10 for
every bond you are short, should your bond holdings be negative.
Likewise, you pay dividends on shares you are short.}

Should you have insufficient money for a stock or bond purchase, the
computer will automatically place an order, in your name, to sell
bonds in a quantity sufficient to raise the money you need.  The
computer will never let your money holdings go negative; negative
money holdings would amount to interest-free loans. 

\section{V. Money, working capital, and the equilibrium interest rate}%
You're probably wondering what fundamental forces will determine the
equilibrium interest rate in this economy.  While a natural first
reaction is consternation and confusion over the apparent
indeterminacy of our interest rates, on second thought there really is
a great deal to go on.

Our first insight is that the interest rate will not be negative; no
one will lend at a negative interest rate because money offers an
interest rate of zero.

Having established that the interest rate will be at least zero
percent, let's consider why it might actually be positive.  The
interest rate is the rental price of money.  Now, who would want to
rent money?  The answer is, anyone who hopes to be a market maker
i.e. earn the bid-ask bounce.  Your friends the market makers will
need a stockpile of money in order to avoid forced sales when their
limit buy orders cross with market sells.  For example, suppose a
market maker has only \$4 money, and her limit buy at \$9 crosses with
a market sell.  The trade drops money to --\$5, which instantly
generates a forced sale on behalf of the market maker.  Result: the
market maker earned the bid-ask spread on the first trade, then lost
it back on the forced sale.%
\note{In fact, it's even worse for the market maker, because the forced
sale will cross with the best limit buy order {\it below} hers on the
order book.}

For a market maker, then, a money stockpile is working capital.  The
next question to ask is, how high an interest rate would a market
maker pay for this working capital?  The market maker will look at
how productive the working capital will be, which means how much
income it will support.  A small amount of working capital---say
\$20---could pretty easily be parlayed into profits approaching 100\%
per day.  A larger amount of working capital will support profits
larger in dollars but smaller as a percent of working capital.
Market makers will look at the interest rate and borrow money up to
the point where the marginal productivity of the money equals the
interest rate.  So, if the interest rate is low, market makers will
borrow a great deal of money; if the interest rate is high they will
borrow less.

That, in other words, sums up the demand for money!  As for the
supply of money, it's just a vertical line (i.e. infinitely
inelastic) at the sum of everyone's initial endowments.  Ultimately,
the interest rate(s) will be whatever clears the ``rental'' market
for money and part of your challenge to develop a feel for that.

\section{VI. How to use the program}%
\subsection{A. Getting in}%
Log into your e-mail account, just as you did to send me your
password.  At the main menu (where you chose ``Pine'' to send me
mail), type a capital $X$.  The screen will go blank except for a
short message like {\it Haas(1)}.  When you see that, type {\it aqc}
and hit return.

From here on, control passes to our game.  It prompts you for your
knickname; type that in.  It then prompts you for your
password---that's the password you e-mailed me last week---type that
in, too.  (For your security, it won't appear on the screen, but
there's a danger in this, too; it's easy to mistype your password, so
if the computer rejects it, try again.)

\subsection{B. How to play}%
Basically, there are two sorts of things you can do in this game.
You can place orders, and you can obtain information.  All your
options are listed for you in two menus.

Use the top-level menu to select which security you'd like to look at
or trade---the stock or the bond.%
\note{A third option in the main menu---`\$'---gives you your money
holdings.  The $f$ option lets you ``finger'' someone, i.e. check on
his or her portfolio.} Once you pick one of the two securities, you
drop into the other menu, which presents you with the following
choices...

\item{b.}Check limit order {\bf b}ook.
Bids are on the left, offers on the right.  They are ordered from
best to worst.  You see the knickname of the person who placed the
order, the number of shares or bonds ordered, and the price.

\item{o.}Place an {\bf o}rder.
This option prompts you for the type of order (market/limit), the
side you want to take (buy/sell), the number of shares or bonds, and
the price (for limit orders).  If you want to trade with the best bid
(offer), place a market order to sell (buy).  A limit order will not
cross with the other side of the order book, regardless of the price
you enter.

\item{c.}{\bf C}ancel one of your orders.
Limit orders, that is; market orders execute as soon as you place them.
Limit orders stay on the limit order book until they cross with a market
order, or you cancel them, or the game ends.

\item{a.}{\bf A}ccount activity.
Here you can see a record of all the orders you placed, and what
became of them---what traded, what didn't, and what you cancelled.
You also see where your holdings of the security stood after each
event.  A {\it C} in the first column indicates a cancelled order. A
{\it F} indicates an forced sale of bonds (which is done for you if
you're short of money).

\item{t.}{\bf T}icker history.
Chronological list of all trades, showing player's knickname, number
of shares or bonds traded (minus sign indicates a sale), price, and
the number of shares or bonds in the person's portfolio after the
trade.

\item{p.}{\bf P}rospectus.
This is a short statement that describes the security and the
determinants of its payoffs.

\item{q.}{\bf Q}uit.
If you want to leave, type {\it logout}, too.  If you want to use
other Haas computer services, type {\it menu}.

\subsection{C. Availability}%
The program is available to you 24 hours a day until the game ends.
You may watch the action as much as you like, but you are limited to
two hours of playing time.  ``Finger'' yourself if you want to know
how much time you've used up.

\section{VII. Grading}%
Do whatever you think will make you rich at the end of the game.
Grading depends on your standing relative to your classmates.  You
start at a B+.  For every 1/2 standard deviation%
\note{Calculated after throwing away the top and bottom 5\% of the data.}
above
(below) the mean you gain (lose) 1/3 of a grade.  For example, if you
finish 1.51 s.d. above the mean you will have an A+.

\section{VIII. Strategy hints}%
First, keep your money invested in stocks or bonds; money by itself
earns no interest.  The problem of how to allocate your money over
stocks and bonds is straightforward in principle: forecast the
dividends yet to come and discount at the interest rate, buying stock
if it's undervalued in the market, selling it if it's overvalued.

That's the classic NPV reasoning you already know about.  Beyond that, there
are a couple of tactical issues that could make a difference: should you
place market or limit orders, and when should you do so?

The choice between market and limit orders presents a tradeoff.  A
market order gets executed instantly, enabling you to capitalize on
some temporary mispricing.  A limit order, on the other hand, could
buy or sell you the security at a better price, but you have no
assurance that a limit order will execute at all.

Timing turns on several considerations.  Certainly, if the DJIA does
something dramatic and you think you're one of the first people to
hear, that's the time to buy stock; there will be limit buy orders
reflecting a now-underestimated DJIA volatility.  Another time you
could catch some people napping is at the NYSE's opening bell (9:30AM
New York time).  The Dow often makes its biggest moves of the day
during the opening half-hour.  OK, maybe you don't feel like getting
up that early.  If so, consider cancelling your limit orders before
you go to bed.

One profitable but risky strategy is to set yourself up as a
marketmaker.  You do this by placing a large bid and a large offer
inside the spread.  If you are successful, then every time someone
places a market order, you are on the other side of the deal and you
repeatedly buy low (at your bid) and sell high (at your offer).  But
beware: there's no guarantee your bids and offers will execute in the
same amounts.

\section{IX. Poisson, the noise trader}%
A fictional character named Poisson (actually a computer program
running out of my account) will place random market orders daily from
8AM to 6PM, while the game is up and running.  Poisson provides an
important element of real markets that would otherwise be missing
from our game---noise trading.  A noise trader (in the real world) is
a person who trades for other than speculative motives, or speculates
but does so in an unintelligent fashion so as to appear random.  The
principal nonspeculative motive for trading is consumption and
saving; people sell securities when they want to finance a major
purchase, and buy securities to save.  Seen in aggregate, the
behavior of these traders appears random, hence the term {\it noise
traders.} Noise traders are an important feature in the real world,
so it's time we simulated them in the game.

Poisson's trading will benefit you.  The benefit is that if you have
a good bid or offer on the limit order book, you can expect it to
cross even if none of your classmates think you're being particularly
generous.  Poisson's average order size is 10 shares or bonds,%
\note{Poisson will be active in both markets.}
alternating randomly between buys and sells, and he trades about
every five minutes.  The process of buying high (at the ask) and
selling low (at the bid) ensures Poisson will lose money to those of
you with the most competitive bids and asks.

Poisson will not trade if the bid-ask spread is wider than 10\% of the
lowest ask; he's random but not stupid.

\section{X. Fed}%
In the practice round, Fed rigged the bond market to support it at a
constant interest rate.  This time Fed's only role will be to make
sure there's {\it something} on the bid side of the bond market at
all times.  That assures that you will always be able to borrow money.
However, Fed's interest rate will be very high.  You have to hope your
classmates will place bids at lower interest rates (i.e. higher prices).

\section{XI. Timetable}%
The precise timing of events is important, so here it is again, in summary.
All times are Pacific coast times.
\medskip
\settabs 2 \columns
\+6PM, Monday July 11	&trading begins.\cr
\smallskip
\+1PM Tuesday--Thursday   &NYSE trading ends, closing DJIA is determined.\cr
\smallskip
\+1PM, Thursday  &Stocks pay dividends.\cr
\+               &Bonds pay \$10.\cr
\+               &Your final wealth is your final money.\cr
\+		 &Game ends.\cr

\bye