\input /users/ted/TeX/workpap.tex
\font\katan=cmr10 
\font\tti=cmitt10 at 12pt 
%***** TITLE PAGE **** 
{\baselineskip=12pt 
\nopagenumbers 
\hrule height 0in 
\vfil 
\centerline{{\bf Learning by Doing:}} 
\centerline{{\bf an Automated Securities Exchange Teaches Finance Theory}} 
\bigskip 
\centerline{by} 
\bigskip 
\centerline{Theodore D. Sternberg} 
\centerline{Haas School of Business} 
\centerline{University of California at Berkeley} 
\centerline{350 Barrows Hall} 
\centerline{Berkeley, California 94720} 
\centerline{internet: sternber@haas.berkeley.edu} 
\bigskip 
\centerline{October, 1993} 
\vfil\vfil 
%***** ABSTRACT PAGE ***** 
\hrule height 0in 
\vfil 
\hrule 
\bigskip 
\centerline{{\bf ABSTRACT}} 
\medskip 
\noindent 
An interactive multiplayer game harnesses two powerful 
forces---students' competitiveness, and their susceptibility to 
video-game addiction---to create a genuine interest in 
finance theory.  The setting is a 24-hour-a-day 
automated securities exchange, in which students endeavor to enrich 
themselves through trades with their classmates. 
\bigskip\hrule 
\vfil\vfil\eject 
} 
 
 
%******** BEGINNING OF MAIN TEXT ****** 
\pageno=1 
\headline{\tenpoint\sl An Automated Securities Exchange\hfil \the\pageno } 
\footline{\hfil} 
\noindent 
Normally, it is difficult to get students interested in finance 
theory.  As a set of statements about imaginary ``perfect markets'' 
worlds, theory has not just its intrinsic difficulty working against 
it, but also students' feelings that it is irrelevant anywhere beyond 
problem sets and exams.  And when they do manage to learn about risk 
premiums, dividend irrelevance, the term structure and all that, 
students seem to gain a mostly mechanistic and uncritical 
appreciation for the ideas we {\it really} wish would become a 
powerful framework for thinking. 
 
In my courses, I have found a way to harness two powerful 
forces---students' competitiveness, and their susceptibility to 
video-game addiction---in the interest of motivating students to 
become interested in finance theory.  The setting is a 24-hour-a-day 
automated securities exchange, in which students endeavor to enrich 
themselves through trades with their classmates.  Quite soon, 
students realize that finance theory is about making money!% 
\note{We don't play for real money.  Rather, 10\% of the 
course grade turns on trading profits.  Students start at a B+ and 
gain or lose 1/3 of a grade for every standard deviation their final 
wealth lies away from the mean.  As an objective function, this 
is both easy to understand and concave.} 
 
It is important to point out what this game is not.  It is not a 
stock-picking contest like the AT\&T Investment Challenge.  It is, 
rather, a self-contained economy in which asset prices are determined 
entirely by the actions of game participants through their orders and 
trades.  The market is therefore only as efficient as students are 
sophisticated about finance and so capturing abnormal profits is very 
much a matter of skill (rather than, as in the real world, of luck.) 
A further advantage is that the set of tradeable securities, i.e. the 
economy, can be chosen and designed so as to focus students' attention 
on a certain set of finance issues.  These issues begin with microstructure 
and range to bond arbitrage, the term structure of interest rates, 
corporate finance, options and futures, and foreign exchange. 
 
Toy securities exchanges are nothing new either (see Forsythe {\it et al} 
(1992)). 
The one described here, however, is unique even among 
real securities exchanges in that it generates endogenous interest rates, 
while handling several currencies with currency-specific interest rates 
to boot, all within a one-stop shop.  All record-keeping, all transaction 
clearing, and all price quotation is automated. 
 
A game that illustrates all the theoretical material of a single 
solid finance course would be too complicated to be fun.  Therefore, 
I break my semester up into a series of these ``market games''.  Each 
game emphasizes a particular set of topics, through the securities 
that we trade.  I start the semester with a very simple setting, and 
build on it.  Section 1 employs a short session transcript to show 
how the game works.  Section 2 describes several of the game settings 
I have used.  Section 3 deals with a few technical issues of 
implementation. 
 
\section{1. How the game works}% 
In rough outline, the game program does just a few things.  It 
accepts limit or market orders, displays a limit order book, clears 
transactions, and displays informational items such as a ``ticker 
tape'' and a chronological record of one's activity showing orders, 
trades and portfolio.  The program is available 24 hours a day, from 
anywhere that one can log into the school's computer.  Thus, students can 
play whenever they please.  There is essentially no limit to the 
number of students who may play simultaneously. 
 
Perhaps the best way to illustrate how the program works is by way of 
a short sample session.  Whatever the computer produces is shown in 
the {\tt typewriter} font.  User input is shown in {\tti italic 
typewriter} font.  Comments appear in the {\katan small normal} font. 
\bigskip\bigskip 
 
\tt 
\parskip=0pt 
\parindent=0pt 
\baselineskip=12pt 
 
{\obeylines\obeyspaces 
tcsh> {\tti aqc} 
{\katan tcsh is the UNIX prompt.  aqc invokes the game program.} 
\smallskip 
 
Enter your knickname: {\tti Diligent} 
Enter your password: 
\medskip 
{\katan Students play under made-up names.} 
 
\bigskip\bigskip 
\vbox{ 
{\katan The first thing Diligent sees is the main menu...} 
\medskip 
     *** main menu *** 
   s.   stock market 
   b.   bond market 
   c.   call option market 
   p.   put option market 
   f.   firm's holdings 
   \$.   money balance history. 
   q.   quit 
Main menu: select sbcpf\$, or q to quit.> {\tti s} 
} 
 
\bigskip\bigskip 
\vbox{ 
{\katan Having selected `s', Diligent drops into the stock menu. (The bond menu looks the same.  Option menus include 'e' for exercise-option.)...} 
\medskip 
      *** stock market *** 
   b.   check limit order Book 
   o.   place an Order. 
   c.   Cancel an order. 
   a.   Account activity 
   t.   Ticker history. 
   p.   Prospectus. 
   x.   eXit 
stock menu: select bocatp, or x to exit.> {\tti b} 
} 
} %%% end of \obeylines\obeyspaces group 
 
\bigskip\bigskip 
\vbox{ 
{\katan Diligent will start by checking the limit order book for stocks.} 
\medskip 
 
 
\centerline{*** limit order book -- stock ***} 
\hfil ***Bids***\hfil\hfil\phantom{xxxxx}***Offers***\hfil 
{\settabs\+\indent&Diligentxxxx&30xxxx&12.99xxxxxxxxxx&Diligentxxxx&30xxxx&12.99\cr %sample line 
\+   &Blazer    &22  &13.10           &Mogul    &10  &13.30 \cr 
\+     &Mogul    &10  &13.00          &Starman    &35  &13.30 \cr 
\+    &Diligent   &15  &12.90         &Msboesky   &10  &13.40 \cr 
\+     &Wagter    &30  &12.90         &Starman    &40  &13.40 \cr 
\+     &Jinxed    &\phantom{0}5  &12.80  &Prudence    &\phantom{0}5    &13.50 \cr 
\+      &Amit    &20  &12.80          &Pagel    &40   &13.90 \cr 
} 
\medskip 
{\katan The format is name --- number of shares ordered --- price. Diligent will now choose option 'o' for place order.} 
} %%% end of vbox 
\medskip 
stock menu: select bocatp, or x to exit.> {\tti o} 
 
\bigskip\bigskip 
{\obeylines\obeyspaces 
{\katan The program responds with some prompts...} 
 
\medskip 
\vbox{ 
Enter M for market order, L for limit order: {\tti m} 
Enter B for buy, S for sell: {\tti b} 
Enter number of units: {\tti 2} 
\medskip 
buy     2 units. 
Is that ok? (y|n) {\tti y} 
} %%% vbox 
\medskip 
{\katan Diligent has placed a market order to buy two shares, and has just confirmed the program's query as to whether that is indeed his intention.} 
 
\bigskip\bigskip 
Execution...diligent  2 13.300 Thu Oct 28 22:00:09 1993 
\medskip 
{\katan Within less than a second, the trade has been executed, and Diligent notified of what he has done---bought two shares at \$13.3 per share (\$13.3 being the best offer on the order book.)} 
} %%% end of \obeylines\obeyspaces group 
 
\bigskip\bigskip 
\vbox{ 
{\katan Let's look at the effect this trade has had on the order book.  The effect is that Mogul's offer has been reduced from 10 shares to 8.}\hfil\break 
stock menu: select bocatp, or x to exit.> {\tti b} 
 
\medskip 
\centerline{*** limit order book -- stock ***} 
\hfil ***Bids***\hfil\hfil\phantom{xxxxx} ***Offers***\hfil 
{\settabs\+\indent&Diligentxxxx&30xxxx&12.99xxxxxxxxxx&Diligentxxxx&30xxxx&12.99\cr %sample line 
\+     &Blazer   &22  &13.10            &Mogul    &\phantom{0}8  &13.30  \cr 
\+      &Mogul   &10  &13.00          &Starman    &35  &13.30 \cr 
\+   &Diligent   &15  &12.90         &Msboesky    &10  &13.40 \cr 
\+     &Wagter   &30  &12.90          &Starman    &40  &13.40 \cr 
\+     &Jinxed   &\phantom{0}5  &12.80   &Prudence &\phantom{0}5  &13.50 \cr 
\+       &Amit   &20  &12.80           &Pagel    &40  &13.90 \cr 
} 
} %%% vbox 
 
{\obeylines\obeyspaces 
\bigskip\bigskip 
{\katan Another place to see the trade's effect is in the ticker display.  The first two lines show the most recent trade.  Subsequent lines show prior trades.  Users can scroll through the whole ticker if they wish.} 
\medskip 
stock menu: select bocatp, or x to exit.> {\tti t} 
} 
\vbox{ 
\medskip 
\centerline{*** ticker history -- stock ***} 
{\settabs\+DiligentXXXXXXXX&-2XXX&13.311XXXXXXXX&77XXXXXXX&\cr 
\+NAME                  &\phantom{00}TRADES  &&\kern-1em HOLDINGS     &TIME\cr 
\+Diligent              &\phantom{-}2  &13.300    &16     &Oct 28 22:00:09\cr 
\+Mogul                 &-2 &13.300    &\phantom{0}8      &Oct 28 22:00:09\cr 
\+Shooter              &\phantom{-}5  &13.300    &81       &Oct 28 21:58:44\cr 
\+Mogul                &-5            &13.300    &10       &Oct 28 21:58:44\cr 
\+Iamastar             &-3  &13.100    &-9       &Oct 28 21:58:10\cr 
\+Blazer               &\phantom{-}3            &13.100    &38        &Oct 28 21:58:10\cr 
} 
} 
\bigskip\bigskip 
{\katan For yet another view of what has happened, Diligent can check 
on his stock account.  Here he sees the entire history of orders he has 
placed, trades that have involved him, and the effects on his holdings.} 
 
\medskip 
stock menu: select bocatp, or x to exit.> {\tti a}\hfil\break 
 
\vbox{ 
\centerline{                *** Diligent -- stock activity *** } 
\settabs\+buyxxxxx&15xx&12.90xxxxxxxxxx&2xx&13.300xxxxxxxxx&10xxxxxxx&Oct 28 21:44:52\cr 
\+ORDERS&    &          &TRADES &         &\kern-1em HOLDINGS  &TIME\cr 
{\katan The first two lines report on the purchase Diligent has just made...} 
\+     &     &          &2 &13.300        &16      &Oct 28 22:00:09\cr 
\+buy  &\phantom{0}2&   &  &              &14      &Oct 28 22:00:09\cr 
{\katan The next line records a limit order placed earlier (it's still in the order book above)...} 
\+buy  &15   &12.90     &  &              &14      &Oct 28 21:44:52\cr 
{\katan Cancelled orders are indicated, as shown below, by a {\tti C} in the first column...} 
\+\kern-6pt Cbuy  &30  &12.70  & &         &14      &Oct 28 17:05:19\cr 
{\katan Lone trades, as shown in the next two lines, indicate the execution of standing limit orders of Diligent's...} 
\+     &     &         &\kern-6pt -4 &13.200 &14   &Oct 28 16:55:01\cr 
\+     &     &         &8 &12.700            &18   &Oct 28 11:20:59\cr 
} 
 
\bigskip\bigskip 
{\katan Every trade, of course, affects Diligent's 
money holdings.  He needs to keep close tabs on these, and to do so he 
exits to the main menu...} 
\medskip 
stock menu: select bocatp, or x to exit.> {\tti x} 
\medskip 
{\katan ...and selects option \$.} 
\medskip 
Main menu: select sbcpf\$, or q to quit.> {$\$$} 
\medskip 
\vbox{ 
\centerline{               *** Diligent -- money *** } 
\settabs\+bondxxxxxxxxx&58xxxxx&8.672xxxxxxxxxx&5.900xxxx&Oct 28 22:02:09\cr 
\+SECURITY       &       &\kern-2em TRADES            &MONEY     &TIME\cr 
\+stock  &\phantom{0}2 &\kern-6pt 13.300    &\kern-6pt 33.300    &Oct 28 22:00:09\cr 
\+stock  &-4 &\kern-6pt 13.200    &\kern-6pt 59.900    &Oct 28 16:55:01\cr 
{\katan The next line shows a forced sale, triggered by a purchase of stock that exhausts Diligent's money holdings.} 
\+bond   &-1     &8.700                     &7.100               &Oct 28 11:20:59\cr 
\+stock  &\phantom{0}8 &\kern-6pt 12.700    &\kern-6pt -1.600    &Oct 28 11:20:59\cr 
\+endowmt&\phantom{0}0    &0.000 &\kern-12pt 100.000    &Oct 19 23:00:00\cr 
} %%% vbox 
\bigskip\bigskip\bigskip 
 
\rm 
\input /users/ted/TeX/workpap.tex
\parskip=4pt 
\parindent=2em 
This concludes the sample session.  Although we have not seen how 
foreign exchange trades are accounted for, or how options are exercised 
or forward contracts delivered, the user interface and display formats 
are similar to those in the examples above. 
 
\section{2. Many possible games}% 
Each of the following subsections describes a different game setting. 
A game setting consists of two things, essentially.  It consists, 
first, of the securities that are traded, together with rules for 
determining payoffs.  The second component is students' initial 
endowments. 
 
While the variety of games is potentially infinite, there are three 
basic setups: (2.1) exogenous interest rate, (2.2) endogenous interest rate, 
and (2.3) multiple currencies. 
 
\section{2.1. Exogenous interest rate}% 
 
\section{2.1.1 The simplest possible setup}% 
This is the easiest version of the game, and it is the one I start with. 
Students are endowed with ten shares of stock and \$100.  Money earns 
the fixed, exogenous interest rate.  Alternatively, students may spend 
their money on stock (or raise more money by selling stock). 
 
Stock pays a dividend at the end of the game (a one-week-long game is 
adequate for this version).  While any stochastically determined 
dividend will do, I have had good results using the Dow Jones 
Industrial Average (DJIA) for my ``random number generator''.  I tell 
students that the one dividend on each share will equal the sum of 
the absolute percentage changes in the DJIA over the five trading 
days coinciding with our game.% 
\note{I do not even begin to instruct 
students in the analytics of conditional sums of random variables, 
or even suggest there {\it could} be an analytical approach.  Instead, I 
encourage students to think heuristically about the forecasting 
problem, and that is what they do. 
The point of this simple version of the 
game is to get students to think about discounting, the time value of 
money, and rudimentary microstructure ideas.} 
 
Set up as described, this game would see very little trading.  As 
Milgrom and Stokey (1982) have argued, nothing much will happen in a 
market where everyone is a speculator.  The market needs some liquidity 
or noise traders, and for that purpose I have a computer-driven noise 
trader known as Poisson.  Students are told about Poisson and his 
habits, which are to place random-size orders at random times.% 
\note{Poisson's order size is normally distributed (and rounded to the 
nearest integer).  A positive order size means {\it buy}, while a negative 
realization of the normal random variable means {\it sell}.  Inter-order 
times are exponentially distributed.  I let students know the parameters 
of this compound Poisson process, explaining in simple terms what those 
parameters imply about the average trade size and the average inter-trade 
interval.} 
 
Poisson loses money as he buys at the lowest ask and sells at the highest 
bid.  Poisson's loss, of course, is everyone else's gain, and students 
learn early on about the pleasures and perils of competing for the 
role of market-maker.% 
\note{As competition closes the bid-ask spread, students 
start to trade with one another to an extent that far overwhelms the 
volume of trading with Poisson. Thus, Poisson only primes the pump. 
Hubris defeats Milgrom and Stokey's no-trade theorem.} 
 
After a week of this game, students are ready for something more challenging 
and complicated.  They have learned how market and limit orders interact; 
they have learned about the tradeoffs between placing a market and a limit 
order; they have learned or reviewed the principles of interest compounding 
over fractional intervals (money holdings compound continuously and students 
will check the program's calculations at least once); they have learned 
that, apart from news, stock prices grow at whatever rate investors used 
to discount future dividends.% 
\note{I have found that most undergraduate and MBA-level students have not 
grasped {\it why} exactly one discounts future cash flows.  The market 
game leads students to ask the right question: do I get a better return 
holding money, or holding stock?  And upon some reflection, they come to 
understand discounting on a deeper level, as essentially an algorithm 
to account for the opportunity cost of not taking an alternative investment.} 
 
 
\section{2.2. Endogenous interest rates}% 
Interest rates are defined, as in the real world, by the prices of bonds. 
Money, which in the {\it exo}genous interest rate game earns steady 
interest, must earn no interest if the bond market is to be relevant. 
The role of money is that bonds and other securities have to be paid for in money.  Moreover, 
the program sees to it that no one's money balance ever becomes negative; 
negative money would amount to an interest-free loan, and that too would 
render the bond market irrelevant.  Thus, whenever someone buys more 
of a security than his money balance allows, the program instantly executes 
a forced sale of bonds on behalf of that person.% 
\note{There is an example of a forced sale near the end of the 
transcript of the sample session, above.} 
When there are 
markets in more than one bond, one of them is designated the ``forced 
sale'' market.% 
\note{Normal trading also takes place in the forced sale market.} 
 
I introduce my students to endogenous interest rates, the term 
structure, forward rates and bond arbitrage all in a two-week economy 
with three coupon bonds.  What ensures that the equilibrium interest 
rate in this economy will be positive is the transactions demand for money 
by everyone who cares to be an active trader (or market-maker).% 
\note{One could, in principle, rely on forced sales to raise 
money as needed.  However, that is a losing strategy as forced sales 
are executed by market order and so incur the bid-ask spread, wiping 
out whatever might have been gained from market-making activities.} 
 
For added challenge, Poisson can be cranked up to deliver money supply 
shocks of any desired magnitude. 
 
\section{2.3. Multiple currencies}% 
A pound bond market, a dollar bond market, and a dollar-pound spot 
exchange market suffice to produce a striking demonstration of the 
monetarist model of exchange rate determination. 
 
In the simplest setting, students are told that at the end of the 
game (one week suffices) their wealth will be calculated as the sum 
of the dollars and pounds they own.  That is, the terminal spot rate 
is fixed at unity. 
 
Students start the game endowed with \$400 and {\it\$}100 (bonds of 
both currencies being in zero net supply).  The unequal money 
supplies ensure the dollar and pound interest rates will be 
different.  That, in turn, produces a spot exchange rate different 
from the terminal exchange rate.  Poisson (the program-driven noise 
trader) leaks both dollars and pounds, thereby affecting money 
supplies randomly, in turn affecting equilibrium interest rates and 
through them the exchange rate. 
 
In more advanced setups, we can trade options on bonds or currencies, 
or bring back the firm and give it earnings in more than one currency. 
Another realistic twist is to free the terminal exchange rate, possibly 
making it equal the ratio of the terminal money supplies with 
money supply shocks delivered by Poisson during the course of the game. 
Forward trading in currencies is also implemented. 
 
From a technical point of view, the foreign currency implementation is 
special in that there has to be one forced-sale market for each currency, 
and each security has to have a currency designated as the currency 
in which its transactions take place. 
 
\section{2.4 Variant games}% 
This section describes some multi-security setups.  These can be played 
with both endogenous and exogenous interest rates, as well as with 
more than one currency. 
 
\section{2.4.1 More securities}% 
The program can handle any number of securities, and two good ones to 
to introduce are a call and a put on the stock.% 
\note{The program can handle American as well as European options. 
Perhaps ironically, I let beginning students play with American options, 
and restrict intermediate-level students to European options.  American 
options are much more fun for investors who fly by the seat of their pants. 
When I really want students to apply option pricing theory, however, 
American options are frustratingly difficult (besides not being subject 
to strict put-call parity arbitrage constraints.) } 
Students learn to think of option pricing and 
put-call parity as a way to make money, not just a potential exam 
question.  They also learn, by means of the fully-visible order book, 
why it is that arbitrage opportunities disappear when exploited. 
 
\section{2.4.2 Corporate finance}% 
In my principles course, I do not move from the basic setup to options, 
but rather to an illustration of corporate finance concepts.  I let the 
game run for two weeks and explain that at the end of the first week, 
we will hold a shareholders meeting. 
 
The agenda for this meeting is to determine the firm's investment, 
dividend and capital structure decision.  For this purpose, the firm's 
earnings have to be defined differently than they were in the basic 
version.  Rather than base the earnings on the sum of the DJIA's 
movements over the week, the shareholders may choose how many ``Dow 
days'' they want to invest in.  One Dow day yields the largest 
one-day absolute percentage change during the week.  Two Dow days 
yield the sum of the two largest one-day changes, and so forth up to 
five Dow days. 
 
Hence the firm faces a decreasing returns to scale production 
function.  I set the price of a Dow day at a level that makes two or 
three days an optimal investment, and leave the rest to the students. 
They decide on the investment, and during the same class session they 
decide on financing and dividends.  Financing is available in the 
debt market (at the exogenous interest rate) and in the stock market. 
To raise money in the stock market the firm places a market order 
right into the same order book everyone else sees. 
 
This describes the basic corporate finance game.  One variant I have used 
has a market for corporate bonds.  The firm enters the world saddled with 
a heavy debt burden, and default is possible.  In one of my classes, we 
held a spirited negotiating session in which shareholders attempted to 
extract a payoff in return for declaring bankruptcy and turning the 
reins over to bondholders. 
 
Corporate finance setups are complex enough to require that the instructor 
devote considerable thought to rules governing permissible shareholder 
actions, bond indentures and short-stock-position covering. 
 
The corporate finance game can, of course, also be played in an economy 
with endogenous interest rates.  Thus, interest rates, stock prices, 
and the firm's investment demand become jointly dependent variables. 
This makes for quite a complex economy, possibly overwhelmingly so for 
all but the most sophisticated and motivated students. 
 
\section{3. Implementation issues.}% 
\section{3.1 Computer considerations}% 
The program currently runs under Digital Equipment Corporation's 
ULTRIX brand of the UNIX operating system, but there should be little 
difficulty in recompiling the program (some 10,000 lines of C) on any other 
UNIX system, especially Berkeley-type UNIX systems such as the Sun OS. 
Porting to other multitasking operating systems would require a small 
amount of work for anyone familiar with interprocess communication 
and signal handling in the C that runs on the operating system in 
question. 
 
The program can also run on a stand-alone PC.  However, all users would 
have to physically be at that one machine to play, and obviously only 
one person could play at a time.  Also, the files used to keep track of 
players' accounts, the order book and everything else would be hard to 
protect from accidental or deliberate destruction. 
 
Demand on system resources is quite small.  Execution of trades 
involves reading and writing a few hundred bytes to a half-dozen 
files.  Other than that, ``playing'' the game (prudently) involves 
looking and thinking much more than placing orders, and that is quite 
inexpensive.  The entire storage requirements for the files generated 
in one week of active play by forty students is about four megabytes. 
The program itself, together with utilities such as the dividend-paying 
program, poisson the program trader, the account creation utility, 
and the server, occupies some 1000K compiled. 
 
\section{3.2 Availability}% 
The author can supply a one-user demo disk that runs on any PC. 
Anyone interested in running the real game is also encouraged to 
contact the author.  
 
\vfil\break 
 
\centerline{\bf REFERENCES} 
\hsize=6 true in 
\hoffset=0.1 in 
\frenchspacing 
\item{1.} Forsythe, R., F.Nelson, G.R.Neumann, and J.Wright, 1992, 
Anatomy of an 
Experimental Stock Market, {\it American Economic Review} {\bf 82}, 
1142-1161. 
 
\item{2.} Milgrom, P. and N. Stokey, 1982, Information, Trade and 
Common Knowledge, {\it Journal of Economic Theory} {\bf 26}, 17-27. 
 
\bye 
